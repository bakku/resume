%%%%%%%%%%%%%%%%%%%%%%%%%%%%%%%%%%%%%%%%%
% Developer CV
% LaTeX Template
% Version 1.0 (28/1/19)
%
% This template originates from:
% http://www.LaTeXTemplates.com
%
% Authors:
% Jan Vorisek (jan@vorisek.me)
% Based on a template by Jan Küster (info@jankuester.com)
% Modified for LaTeX Templates by Vel (vel@LaTeXTemplates.com)
%
% License:
% The MIT License (see included LICENSE file)
%
%%%%%%%%%%%%%%%%%%%%%%%%%%%%%%%%%%%%%%%%%

%----------------------------------------------------------------------------------------
%	PACKAGES AND OTHER DOCUMENT CONFIGURATIONS
%----------------------------------------------------------------------------------------

\documentclass[9pt]{../developercv} % Default font size, values from 8-12pt are recommended

%----------------------------------------------------------------------------------------

\begin{document}

%----------------------------------------------------------------------------------------
%	TITLE AND CONTACT INFORMATION
%----------------------------------------------------------------------------------------

\begin{minipage}[t]{0.45\textwidth} % 45% of the page width for name
	\vspace{-\baselineskip} % Required for vertically aligning minipages
	
	% If your name is very short, use just one of the lines below
	% If your name is very long, reduce the font size or make the minipage wider and reduce the others proportionately
	\colorbox{black}{{\HUGE\textcolor{white}{\textbf{\MakeUppercase{Christian}}}}} % First name
	
	\colorbox{black}{{\HUGE\textcolor{white}{\textbf{\MakeUppercase{Paling}}}}} % Last name
	
	\vspace{6pt}
	
	{\huge Software Entwickler} % Career or current job title
\end{minipage}
\begin{minipage}[t]{0.25\textwidth} % 25% of the page width for the first row of icons
	\vspace{-\baselineskip} % Required for vertically aligning minipages
	
	% The first parameter is the FontAwesome icon name, the second is the box size and the third is the text
	% Other icons can be found by referring to fontawesome.pdf (supplied with the template) and using the word after \fa in the command for the icon you want
        \icon{Gift}{12}{15.05.1995}\\
	\icon{MapMarker}{12}{Würzburg}\\
	\icon{Phone}{12}{+49 173 5680048}\\
\end{minipage}
\begin{minipage}[t]{0.3\textwidth} % 30% of the page width for the second row of icons
	\vspace{-\baselineskip} % Required for vertically aligning minipages
	
	% The first parameter is the FontAwesome icon name, the second is the box size and the third is the text
	% Other icons can be found by referring to fontawesome.pdf (supplied with the template) and using the word after \fa in the command for the icon you want
	\icon{Github}{12}{\href{https://github.com/bakku}{github.com/bakku}}\\
	\icon{Medium}{12}{\href{https://medium.com/@bakku1505}{medium.com/@bakku1505}}\\
	\icon{At}{12}{\href{mailto:christian.paling@googlemail.com}{christian.paling@gmail.com}}\\	
\end{minipage}

\vspace{0.5cm}

%----------------------------------------------------------------------------------------
%	INTRODUCTION, SKILLS AND TECHNOLOGIES
%----------------------------------------------------------------------------------------

\cvsect{Profil}

\begin{minipage}[t]{0.4\textwidth} % 40% of the page width for the introduction text
	\vspace{-\baselineskip} % Required for vertically aligning minipages
        Ich biete ein breites Spektrum an Erfahrung als Software Entwickler, habe jedoch meinen primären
        Fokus im Bereich der Backend Entwicklung und sekundären Fokus in der Frontend Entwicklung.
        Während meiner Zeit als Angestellter und Freiberufler habe ich mehrere Dutzend
        Softwaresysteme neuentwickelt, gewartet, verbessert und skaliert. Bei meiner Arbeit wähle ich
        passende und moderne Technologien für den jeweiligen Anwendungszweck, sowie erprobte Entwicklungsmethodiken
        um hochqualitative Software zu entwickeln.
\end{minipage}
\hfill % Whitespace between
\begin{minipage}[t]{0.55\textwidth} % 50% of the page for the skills bar chart
	\vspace{-\baselineskip} % Required for vertically aligning minipages
	\begin{barchart}{5.5}
		\baritem{Ruby}{80}
		\baritem{JVM (Java/Scala/Clojure)}{60}
		\baritem{Golang}{60}
                \baritem{SQL/NoSQL Datenbanken}{60}
		\baritem{JavaScript}{50}
		\baritem{CSS}{50}
	\end{barchart}
\end{minipage}

%----------------------------------------------------------------------------------------
%	EDUCATION
%----------------------------------------------------------------------------------------

\cvsect{Ausbildung}

\begin{entrylist}
	\entry
		{2019 -- 2021}
		{Master of Science}
		{Ostbayerische Technische Hochschule - Regensburg}
		{Schwerpunkte: Moderne theoretische und praktische Softwaretechniken, Spezialisierte Algorithmen, Verteilte Systeme\\Masterarbeit: Entwicklung von domänenspezifischen Sprachen und polyglotten Applikationen mit der GraalVM\\Endnote: noch ausstehend}
	\entry
		{2013 -- 2017}
		{Bachelor of Engineering}
		{Hochschule für angewandte Wissenschaften - Würzburg}
		{Schwerpunkte: IT-Security, Microservices, Mobile Applikationen\\Bachelorarbeit: Hochskalierbare 12-factor Apps mit Ruby on Rails, Laravel und Play\\Endnote: 1,5\\Auszeichnung als einer der drei besten Absolventen (Professor Wolfgang Maria Fischer Stiftung)}
\end{entrylist}

%----------------------------------------------------------------------------------------
%	EXPERIENCE
%----------------------------------------------------------------------------------------

\cvsect{Praktische Erfahrung}

\begin{entrylist}
	\entry
		{seit 2016}
		{Freiberuflicher Software Entwickler}
		{}
		{- Softwareentwicklung jeglicher Art (Web, Client, Kommandozeilenprogramme, ...)\\
                 \texttt{Ruby}\slashsep\texttt{JVM}\slashsep\texttt{Visual Basic}\slashsep\texttt{Golang}\slashsep\texttt{Docker}}
	\entry
		{2017 -- 2019}
		{Web Developer}
		{Streetspotr Gmbh - Nürnberg}
		{- Entwickeln neuer Systeme der SOA der Firma (Beispiel: Plattform für Datenanalyse für Kunden)\\
                 - Umstrukturierung und Skalierung existierender Web Services mittels AWS\\
                 - Integration neuer Team-Mitglieder bei der Fusion mit der POSpulse GmbH\\
                 \texttt{Ruby}\slashsep\texttt{Golang}\slashsep\texttt{MySQL}\slashsep\texttt{MongoDB}\slashsep\texttt{React}}
	\entry
		{2015 -- 2016}
		{Praktikant Software Entwicklung}
		{4tiitoo Gmbh - München}
		{- Entwicklung eines Web Shops worüber Kunden Plugins für das Hauptprodukt der Firma kaufen\\
                 - Entwicklung von Plugins für das Hauptprodukt der Firma\\
                 \texttt{Ruby}\slashsep\texttt{React}\slashsep\texttt{PostgreSQL}\slashsep\texttt{Elasticsearch}\slashsep\texttt{C++}}
	\entry
		{2014 -- 2015}
		{Werksstudent Quality Assurance}
		{iisy AG - Rimpar}
		{- Schreiben von Integrations- und Akzeptanztests\\
                 \texttt{Java}\slashsep\texttt{Selenium}}
\end{entrylist}

%----------------------------------------------------------------------------------------
%	ADDITIONAL INFORMATION
%----------------------------------------------------------------------------------------

\begin{minipage}[t]{0.45\textwidth}
	\vspace{-\baselineskip} % Required for vertically aligning minipages

	\cvsect{Sprachen}
	
	\textbf{Deutsch} - Muttersprache\\
	\textbf{Englisch} - Fließend
\end{minipage}
\hfill
\begin{minipage}[t]{0.45\textwidth}
	\vspace{-\baselineskip} % Required for vertically aligning minipages
	
	\cvsect{Methodiken}
	
        - Funktionale Programmierung\\
        - Test-Driven Development\\
        - Container- und Cloudbasierte Applikationen\\
        - Agile Entwicklung\\
\end{minipage}

%----------------------------------------------------------------------------------------

\end{document}
